% !TeX root = ../main.tex
% Add the above to each chapter to make compiling the PDF easier in some editors.

\chapter{Introduction}\label{chapter:introduction}
Classical Computing on CPU is not exhaustive. That's why scientists introduce coprocessors such as GPUs and FPGAs.
They were created to accelerate certain types of tasks, which they successfully did, 
but now they have reached the limit according to computability theory.
Thus, there are a lot of problems that cannot be efficiently solved only by changing and adapting classical CPU architecture.
Some of those haven't any proven polynomial classical algorithm.
Nevertheless, a number of quantum states saved in one quantum register grows exponentially with number of qubits, 
that allows to solve some of the problems in efficiently way.
Theoretically, this will give such a scale, which is often called quantum advantage.

Nowadays \ac{QC} don't outperform classical ones, however they have a huge potential to do that in the near future. 
Current state is called \ac{NISQ} which allows exploring a world of quantum entanglements, 
superpositions, error corrections and quantum information interpretation.
To leave this era and go into \ac{FTQC} new ways to correct or not to make errors, 
to improve coherence time and operating temperatures, together with overall fidelity and speed improvements considered.

One of those ways is to explore different architectures for \ac{QC}, which potentially can solve one or more of above-mentioned constraints.
There are already few promising architectures such as Trapped-Ion \ac{QC}, Superconducting \ac{QC}, Topological \ac{QC} and a few more.
In this work only a \ac{NAQC} will be considered. 

\ac{NAQC} has several advantages over other architectures.
For example a long coherence time of qubits that are based on Rydberg states of Rubidium or Cesium.
This allows to make a lot more calculations before fidelity became unacceptable low. 
Moreover, \ac{NAQC} has a rare feature that sets it apart from others.
This feature is a possibility to execute gates not in strictly defined positions but everywhere where a Rydberg laser can point.
Due to this, gates can be applied to several qubits at once implementing a built-in multi-qubit gate support.
And the most interesting now is that atoms can be physically moved by \ac{AOD} 
that can be used for separating zones or as another possibility 
to perform an analogous for SWAP but physically and not logically or moreover consider a combination of those

All this brings a huge number of possibilities for optimizing quantum computations in \ac{NISQ}
and at the same time complexities during compilation.

In this work a different tool chains in compilation will be considered, such as HybridMapper from \ac{MQT}, 
DasAtom or Enola from \ac{UCLA-VAST}.
They will be benchmarked, evaluated, then the issues in calculations will be fixed and the result will be evaluated again.
For consistency the used architecture will be approximated as closely as possible between used compilers
\parencite{Henriet_2020, Schmid_2024_NeutralAtomBasics}.