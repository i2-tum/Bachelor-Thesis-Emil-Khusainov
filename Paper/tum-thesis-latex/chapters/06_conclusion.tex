% !TeX root = ../main.tex
% Add the above to each chapter to make compiling the PDF easier in some editors.

\chapter{Conclusion And Outlook}\label{chapter:conclusion}
\section{Conclusion}
\ac{NAQC} represent a new hardware architecture with exclusive combination of capability and constraints.
This architecture has huge perspectives in world of quantum computations. 
Nevertheless, to subjugate all the power of it a good software solution must be produced.

This work was focused on benchmarking of existing compilation toolchains 
and the most important thing is an honest comparison. What was fullified in this paper.

Firstly, a test setup was created, to bring all tool chains into similar conditions, and first testing was performed.

Secondly, results were interpreted, and suspicious things were investigates and repaired.

Thirdly, the new testing was performed, that has shown a worthy results 
such as finding that difference is actually not a 400x times and exponential grow as DasAtom \parencite{huang2025dasatomdivideandshuttleatomapproach} states, 
but either 5.5x times between Enola and DasAtom. 
\section{Outlook}
As a result, some further work could be done. Something such as adding new compiler toolchains for comparison.
Or something deeper, 
such as consideration of all possible parameters that take place for example in Enola but don't in other,
and due to not very big impact on calculations, often it is not being considered.
Moreover, different circuits could be used, with different level of start optimizations to see actually performance of a tool chain.