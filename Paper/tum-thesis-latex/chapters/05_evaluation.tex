% !TeX root = ../main.tex
% Add the above to each chapter to make compiling the PDF easier in some editors.

\chapter{Evaluation}\label{chapter:evaluation}
For benchmarking of above-mentioned compilation tool chains a python project was introduced \parencite{Emil_Khusainov_Bachelor_GIT}.
The first idea was to compare these three tool chains with architectures as close as possible.
\section{Implementation and testing nuances}
For that purposes the following program structure were made \ref{fig:overview}.
As Test Cases some \ac{QFT} circuits used.
\begin{figure}[htbp]
  \centering
    \includegraphics[width=1.0\textwidth]{figures/schema.pdf}
    \caption{Workflow of benchmarking script}
    \label{fig:overview}
\end{figure}
Since all of these compilers are not actually a compiler in the full sense, but rather tool chains with different tools
a lot of non-trivial architecture adaptations, pre-processing, mid-processing and post-processing steps were required. 

\section{First Evaluation}
Here are main architecture parameters for first evaluation \ref{tab:architecture_first}. 
Then they were  adapted as near as possible to pass into each compiler's own input form, due to it small inaccuracies possible.
\begin{table}[htpb]
  \caption[Architecture First Run]{Architecture parameters for first run}\label{tab:architecture_first}
  \centering
  \begin{tabular}{l l l l}
    \toprule
      Parameter & Value \\
    \midrule
      Interaction Radius & 2 \\
      Rydberg Blockade & 2 \\
      Two Qubit Time & 0.36 \\
      One Qubit Time & 0.36 \\
      Two Qubit Fidelity & 0.9999 \\
      One Qubit Fidelity & 0.9999 \\
      Coherence Time $\mu$s & 1500000 \\
      AOD Activate Time $\mu$s & 20 \\
      Move Fidelity  & 0.9999 \\
      Move Speed $\mu$m/$\mu$s & 0.55 \\
      SLM AOD separation & 2\\
    \bottomrule
  \end{tabular}
\end{table}

The following results were obtained \ref{fig:AllGateCountArch1},
\ref{fig:CZGateCountArch1}, \ref{fig:FidelityArch1}. 
and due to very long compilation time Enola was tested separately in \ac{QFT}30 \ref{tab:EnolaDasAtomQFT30}.
\begin{figure}[htbp]
  \centering
    \includegraphics[width=0.8\textwidth]{figures/AllGateCountArch1.png}
    \caption[All Gate Number of first Architecture]{Number of All Gates in compiled Circuit in first architecture}
    \label{fig:AllGateCountArch1}
\end{figure}
\begin{figure}[htbp]
  \centering
    \includegraphics[width=0.8\textwidth]{figures/CZGateCountArch1.png}
    \caption[CZ Gate Number for first Architecture]{Number of CZ Gates in compiled Circuit in first architecture}
    \label{fig:CZGateCountArch1}
\end{figure}
\begin{figure}[htbp]
  \centering
    \includegraphics[width=0.8\textwidth]{figures/FidelityArch1.png}
    \caption[Fidelity in first Architecture]{Fidelity in first Architecture}
    \label{fig:FidelityArch1}
\end{figure}
\begin{table}[htpb]
  \caption[Enola DasAtom First Run QFT30]{Enola DasAtom First Run QFT30}\label{tab:EnolaDasAtomQFT30}
  \centering
  \begin{tabular}{l l l l}
    \toprule
      Output & Enola QFT30 & DasAtom QFT30 \\
    \midrule
      Fidelity Overall & 0.0008991 & 0.7060\\
      Fid. Movement & 0.69376 & 0.9934373\\
      Fid. Coherence & 0.00154 & 0.81603\\
      Gate Count & 2370 & 4111\\
      CZ Gates & 915 & 915\\
      Fid. 1Q & 1 & 1\\
      Compile Time s & 14251 & 2.5\\
    \bottomrule
  \end{tabular}
\end{table}

\subsection{Result Interpretation}
Firstly, one can observe \ref{fig:AllGateCountArch1} that number of used gates by \ac{HM} with only Shuttling strategy is the lowest, 
when other tools require much more gates. This is because \ac{HM} in shuttling mode only make moves and don't change a circuit,
when other actively changing circuit and \ac{HM} with allowed SWAP uses itself a CZ with Hadamard gates to solve a coupling problem.

Secondly, one can see in \ref{fig:AllGateCountArch1}, in \ref{tab:EnolaDasAtomQFT30}, 
and in \ref{fig:CZGateCountArch1} that Enola uses much fewer one-qubit Gates than DasAtom. 
But when one sees an output of compilation, 
then a fidelity of one-qubit gates is equal to 1.0 
since both Enola and DasAtom doesn't consider impact of one-qubit gates onto fidelity.
Taking into account that DasAtom states 415x times more fidelity than Enola, 
but in same time uses more and not consider one-qubit gates. What is a bit unfair.
For example, for fidelity of one-qubit gates equal 0.9999: \[0.9999^{4111 - 2370} \approx 0.84\]
What would bring a valuable impact on overall fidelity.

Thirdly consider \ref{fig:FidelityArch1} and \ref{tab:EnolaDasAtomQFT30}, \ac{QFT} is not a very large circuit, and parameters for physical \ac{AOD} and \ac{SLM} grid were a very similar.
It would be interesting to know why there is such a difference in move fidelity and coherence fidelity between Enola and DasAtom.
Recap that DasAtom states a huge outperform of Enola and according to results of first run fidelity difference were around 700x times, what is very similar in inaccuracy with 414x times.

Fourthly, it is notable that optimization of DasAtom achieve the same fidelity as Swap Based \ac{HM}, when DasAtom uses only shuttling and \ac{HM} with shuttling is much worse.

\section{Justification and correction of differences}
To find out the reason for such a strong difference a source codes of corresponding gits 
of Enola \parencite{Tan_2025_Enola} and DasAtom \parencite{huang2025dasatomdivideandshuttleatomapproach} were investigated.

\subsection{Single Qubit Fidelity}
As was said earlier, DasAtom and Enola doesn't consider one-qubit gates into fidelity calculation.
Nevertheless, Enola in its paper  has added a one-qubit fidelity calculation, but in source code it was always equal to 1.0 and untouched.
That's why such functionality to both compilers was added with simple formula: \[fidelity_{\mathrm{1Q}}^{N_{\mathrm{1Q}}}\]

\subsection{Investigating Different Coherence Fidelity}
In Enola according to source codes, and it's paper following formula for fidelity coherence is used:
\[\prod_{q \in Q} \left(1 - \frac{t_q}{T_2} \right)\]
It is a first order Taylor expansion and pretty correct, but better to use an exponential variation and also that is used in DasAtom:
\[e^{-t / T_2}\]

So now Enola calculates coherence fidelity with: \[\prod_{q \in Q} \left(e^{-t_q / T_2} \right)\]

\subsection{Investigating Different Move Idle Time for Coherence}
During improving of Enola behavior, was noticed, that an idle time t is very different for Enola compared to the other two.
The search revealed the reason for this difference. DasAtom and \ac{HM} use simple linear formula to calculate time for movement: \[distance/speed\]
Nevertheless, Enola used an approach described by Dolev Bluvstein to calculate move time:
\[200 \sqrt{\frac{distance}{110}}\]
This approach doesn't consider different possible architectures and therefore speed and distances. 
It was created to show a non-linear dependency between time and distance e.g. due to acceleration.
But when distance is not equal to 110 then a difference between resulting times of approaches grows drastically.
That's why for honest testing Enola will use also linear approach.

\subsection{Investigating Different Move Distances}
The following issues were noticed during repair of move time calculation. 
Move distance was very different between DasAtom and Enola. 
For example, when average movement on DasAtom was 11 $\mu$m, on Enola it was more than 200 $\mu$m. 
That was strange and needed further investigation.

As a result a significant source code error was found. 
The Enola compiler doesn't consider architecture parameters on mapping and routing steps, only on scheduling.
Therefore, from the outside it looked like there was some kind of reaction to the architecture changes.
But in mapping and routing step the architecture parameters were defined as global variables, 
when a Set function doesn't consider those as global, but creates local variables with same name.
This mistake was fixed and know it is time to see difference.

\section{Second Evaluation}
Here are a bit new architecture parameters due to achieving more experience with work with these toolchains.
Changed was an \ac{AOD} activate time from 20 $\mu$s to 0.55 $\mu$s due to the diverse utilizations of this parameter,
and therefore different impact on similar \ac{AOD} sequences.
Also, fidelity of two-qubit gates were downgraded from 0.9999 to 0.9996 
due to added consideration of one-qubit gates with fidelity 0.9999,
and obvious that two-qubit gates should have then lower fidelity.
\begin{table}[htpb]
  \caption[Architecture Second Run]{Architecture parameters for second run}\label{tab:architecture_second}
  \centering
  \begin{tabular}{l l l l}
    \toprule
      Parameter & Value \\
    \midrule
      Interaction Radius & 2 \\
      Rydberg Blockade & 2 \\
      Two Qubit Time & 0.36 \\
      One Qubit Time & 0.36 \\
      Two Qubit Fidelity & 0.9996 \\
      One Qubit Fidelity & 0.9999 \\
      Coherence Time $\mu$s & 1500000 \\
      AOD Activate Time $\mu$s & 0.55 \\
      Move Fidelity  & 0.9999 \\
      Move Speed $\mu$m/$\mu$s & 0.55 \\
      SLM AOD separation & 2\\
    \bottomrule
  \end{tabular}
\end{table}

The following results were obtained \ref{fig:AllGateCountArch2},
\ref{fig:CZGateCountArch2}, \ref{fig:FidelityArch2}.
\begin{figure}[htbp]
  \centering
    \includegraphics[width=0.8\textwidth]{figures/AllGateCountArch2.png}
    \caption[All Gate Number of second Architecture]{Number of All Gates in compiled Circuit in second architecture}
    \label{fig:AllGateCountArch2}
\end{figure}
\begin{figure}[htbp]
  \centering
    \includegraphics[width=0.8\textwidth]{figures/CZGateCountArch2.png}
    \caption[CZ Gate Number for first Architecture]{Number of CZ Gates in compiled Circuit in second architecture}
    \label{fig:CZGateCountArch2}
\end{figure}
\begin{figure}[htbp]
  \centering
    \includegraphics[width=0.8\textwidth]{figures/FidelityArch2.png}
    \caption[Fidelity in second Architecture]{Fidelity in second Architecture}
    \label{fig:FidelityArch2}
\end{figure}

\subsection{Result Interpretation}
Firstly, one can observe in \ref{fig:FidelityArch2} that difference between DasAtom and Enola for \ac{QFT}30 is only about 5.5 times higher,
and an exponentially increasing gap is not observed.
Moreover, Enola consider a lot more fidelity affecting factors that were not considered here and not investigated, 
but have not a small effect on fidelity. 
Therefore, it is most likely that by fair consideration of all co-factors the result difference will be a lot smaller than 5.5 times.

Secondly, \ref{fig:FidelityArch2} confirms that Enola is a bit better than shuttling based \ac{HM}, due to missing optimization step in \ac{HM}
and DasAtom could achieve a SWAP level fidelity by using only shuttling and sometimes outperform \ac{HM} with enabled SWAP.
Nevertheless, one should not forget that an input QFT is a bit optimized already by input and maybe on random circuit the results will be different.
