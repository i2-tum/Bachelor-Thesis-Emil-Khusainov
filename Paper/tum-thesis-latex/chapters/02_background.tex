% !TeX root = ../main.tex
% Add the above to each chapter to make compiling the PDF easier in some editors.

\chapter{Background}\label{chapter:background}
\section{Quantum Basics}
\subsection{Idea of Quantum Computation}
For more than 50 years of using classical paradigms of computations it has been recognized, 
that parallel to classical exists also a quantum version of that.
Which offers clearly different and maybe much more powerful features than common computational theory
\parencite{Barenco_1995}.

\subsection{Quantum State and its Representation}
Quantum mechanics uses operations with states in a complex Hilbert space to perform computations.
This quantum state serves as an analog of an ordinary bit in the quantum world.
In the following considered only one-qubit system for simplicity w.l.o.g. in so-called computational basis.
Moreover, this state possesses not only basis states \(|0\rangle\) and \(|1\rangle\),
but also an infinite amount of combinations of this eigenstates, this is called a quantum superposition. 
To represent it two complex numbers \(\alpha\) and \(\beta\) are used.
For one-qubit system we define a quantum state as follows:
\parencite{nielsen00}.
\[
|\psi\rangle = \alpha|0\rangle + \beta|1\rangle
\]
The coefficients must satisfy the following normalization condition:
\[
|\alpha|^2 + |\beta|^2 = 1
\]
To transfer quantum information into classical bits a measurement operation is performed.
Measurement will always collapse a wave function into one of corresponding to result the basis states. 
The probability of outcome is calculated with Born's rule as follows:
\[
P(0) = |\alpha|^2, \quad P(1) = |\beta|^2
\]
For visualization, the state can be represented geometrically as vector in Bloch Sphere, 
where angles \(\theta\) and \(\phi\) will uniquely determine the position of the vector. 
\[
|\psi\rangle = \cos\frac{\theta}{2}|0\rangle + e^{i\phi}\sin\frac{\theta}{2}|1\rangle
\]

%\[|\psi\rangle \otimes |0\rangle \rightarrow |\psi\rangle \otimes |\psi\rangle \]

%\[|\Phi^+\rangle = \frac{|00\rangle + |11\rangle}{\sqrt{2}}\]

\subsection{Quantum Gates}
